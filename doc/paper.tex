\documentclass[11pt,a4paper]{article}
\usepackage{lac2018}
\sloppy
\newenvironment{contentsmall}{\small}

\usepackage{listings}
\lstset{breaklines=true}
\lstset{language=C}
\lstset{basicstyle=\small\ttfamily}
\lstset{extendedchars=true}
\lstset{tabsize=1}
\lstset{columns=fixed}
\lstset{showspaces=false}
\lstset{showstringspaces=true}
\lstset{numbers=left}
\lstset{xleftmargin=2em}  
\lstset{stepnumber=1}
\lstset{numbersep=5pt}
\lstset{numberstyle=\footnotesize\ttfamily}
\lstset{captionpos=t}
\lstset{breaklines=true}
\lstset{breakatwhitespace=true}
\lstset{breakautoindent=true}
\lstset{linewidth=\columnwidth}

\usepackage[hidelinks]{hyperref}

\title{Live-coding Audio in C}

%see lac2018.sty for how to format multiple authors!
\author
{Claude HEILAND-ALLEN
\\ unaffiliated
\\ claude@mathr.co.uk
}

\begin{document}
\maketitle

\begin{abstract}
\begin{contentsmall}

  An audio live-coding skeleton is presented, implemented in C and adaptable to
  languages that expose the C ABI from shared libraries.  The aim is to support
  a two-phase edit-commit coding cycle allowing long-lived signal processing
  graphs to be modified.

  The skeleton watches a directory for changes, recompiling changed sources and
  reloading changed shared objects. The shared object defines an audio
  processing callback. The skeleton maintains a memory region between reloads,
  allowing callback state to be preserved so that audio processing can continue
  uninterrupted.

\end{contentsmall}
\end{abstract}

\keywords{
\begin{contentsmall}
live-coding, audio processing, code reloading
\end{contentsmall}
}

\section{Motivation}\label{motivation}

(At least) two popular music software packages allow live-coding.
Pure-data~\cite{PD} allows the manipulation of audio processing graphs at runtime,
but live-coding requires careful preparation as there is no edit/commit
cycle: every change is live as soon as it is made.  Moreover each edit
recompiles the whole DSP chain, which is not a realtime-safe operation
and can lead to audio dropouts.

Supercollider~\cite{SC3} takes a
different approach, running selected code only when a certain key
combination is pressed.  However, Supercollider DSP code runs on a server
distinct from its language, and synths (compiled by a client such as
\lstinline!sclang! and sent to the \lstinline!scsynth! server) can't
be modified once they are running, beyond setting control variables
or re-ordering synths within a group -- the code within each synth is
immutable.

The main goal of Clive is to be able to edit digital synthesis algorithms
while they are running, with a two phase edit/commit cycle: code can be
edited freely, until it is saved to disk, at which point it is re-compiled
and (if successful) re-loaded into an active audio processing engine.  The
reasons for using the C programming language are familiarity to the author,
precise control of memory layout of data structures allowing code reload
with state preservation, speed of compilation (relative to higher-level
languages like C++), and predictable runtime performance.

Performance with Clive typically involves pre-preparation, at least of
common unit generators like oscillators and filters, perhaps higher-level
effects like chorus or dynamic range compression, extending to more-complete
compositions with complex data flow.  The live-coding aspect involves editing
a file in the performer's favourite text editor, with the act of saving with
Ctrl-S or other shortcut being timed to allow the new code to start executing
after the latency of compilation.

\section{Implementation}\label{implementation}

\subsection{The skeleton}\label{the-skeleton}

The live engine starts a JACK client, whose audio processing function is
a thin wrapper around a \lstinline!callback! function pointer. The
function pointer is initially set to a function that fills its output
buffers with silence. The main program watches the current directory for
changes, recompiling changed sources and reloading a shared library
(named \lstinline!go.so!) on changes. The shared library defines a
symbol \lstinline!go! matching the \lstinline!callback_t! interface. The
callback is set to the address of this function. The callback is passed
a pointer to a memory area that is preserved between code reloads.

\begin{lstlisting}[language=C, caption={The skeleton}, label=lskeleton]
typedef void (*callback_t)(void *, float *, int, float *, int);
volatile callback_t callback = silence_callback;
void *data;
int process(jack_nframes_t n, void *arg) {
 float in[nin];
 float out[nout];
 // ... get JACK buffers
 for (int i = 0; i < n; ++i) {
  // ... copy buffers to in
  callback(data, in, nin, out, nout);
  // ... copy out to buffers
 }
 return 0;
}
int main() {
 data = calloc(1, bytes);
 load_library("go.so");
 // ... launch JACK client
 // the following is pseudo-code
 watch w = watch_directory(".");
 event e;
 while (( e = wait_event(w) )) {
  if (has_changed(e, "go.so"))
   load_library("go.so");
  if (has_changed(e, "go.c"))
   recompile("go.c");
 }
 return 0;
}
\end{lstlisting}

\subsection{Dynamically reload
libraries}\label{dynamically-reload-libraries}

The POSIX.1-2001 \lstinline!dl! interface is used to load shared libraries
at runtime~\cite{DL}. The library is loaded with \lstinline!dlopen!, the
callback function symbol address is found with \lstinline!dlsym!, and when
no longer needed the code can be unloaded with \lstinline!dlclose!.  The type
cast gymnastics on line 9 are to avoid a warning in strict ISO C99 mode.

\begin{lstlisting}[language=C, caption={Using libdl}, label=llibdl]
void *old_library = 0;
void *new_library = 0;
void load_library(const char *name) {
 if (! name)
  return;
 if (! (new_library = dlopen(name, RTLD_NOW)))
  return;
 callback_t *new_callback;
 *(void **) (&new_callback) = dlsym(new_library, "go");
 if (new_callback) {
  set_callback(new_callback);
  // unload old library
  if (old_library)
   dlclose(old_library);
  old_library = new_library;
  new_library = 0;
 } else {
  dlclose(new_library);
  new_library = 0;
 }
}
\end{lstlisting}

\subsection{Mitigate races}\label{mitigate-races}

There is a race condition, especially problematic on multi-core
architectures: the old code must not be unloaded while the JACK callback
is still running it, otherwise the process will crash. To mitigate this
risk (though not prevent it entirely), a flag is added that makes the
main process spin until the current DSP block's JACK processing is
finished. The old library is probably safe to unload after
\lstinline!set_callback! has returned. To avoid a total stop to the live
performace in the unlikely event of this crash, a shell wrapper script
relaunches the engine automatically.

\begin{lstlisting}[language=C, caption={Race mitigation}, label=lrace]
volatile bool processing = false;
int process(jack_nframes_t n, void *arg) {
 processing = true;
 // ...
 processing = false;
 return 0;
}
void set_callback(callback_t new_callback) {
 while (processing)
  ;
 callback = new_callback;
}
\end{lstlisting}

\subsection{Defeat caches}\label{defeat-caches}

Unfortunately \lstinline!libdl! performs rather aggressive caching,
meaning that a library of the same name is not reloaded if it is already
open, even if its contents have changed.

A first try was to unload the old code and load the new code, hoping
that it would be quick enough to avoid audio dropouts (the callback
function pointer was set to silence before unloading to avoid a crash).
The dropouts were too severe to make this viable in a performance
situation.

A second attempt was to use two symlinks for double-buffering, but even
this does not defeat the cache. A real file copy appears to be
necessary, and finally the current version copies alternately to two
separate files in a double-buffering scheme.

\begin{lstlisting}[language=C, caption={Defeating caches by copying}, label=lcache]
const char *copy[2] =
 { "cp -f ./go.so ./go.a.so"
 , "cp -f ./go.so ./go.b.so" };
const char *library[2] =
 { "./go.a.so", "./go.b.so" };
int which = 0;
void copy_library() {
 if (! system(copy[which]))
  return library[which];
 else
  return 0;
}
void load_library(const char *name) {
 // ...
 if (new_callback) {
  which = 1 - which;
  // ...
 }
}
int main() {
 // ...
  if (has_changed(e, "go.so"))
   load_library(copy_library());
}
\end{lstlisting}

\subsection{Notify reloads}\label{notify-reloads}

It can be useful when live programming audio to perform initialization
once only. To that end, clive mandates that the first word of the memory
area is an \lstinline!int!, that is set to 1 when the code is reloaded.
The callback code is responsible for clearing it, necessary to detect
subsequent reloads.

\begin{lstlisting}[language=C, caption={Reload notification}, label=lreload]
volatile bool reloaded = false;
int process(jack_nframes_t n, void *arg) {
 if (reloaded) {
  int *p = data;
  *p = 1;
  reloaded = 0;
 }
 // ...
 return 0;
}
void set_callback(callback_t new_callback) {
 // ...
 reloaded = true;
}
\end{lstlisting}

\subsection{Watch for changes}\label{watch-for-changes}

On Linux, the filesystem can be monitored for changes using
\lstinline!inotify!~\cite{INOTIFY}. The function
\lstinline!inotify_add_watch! is used to check for file close events, of
files that were opened for writing. The \lstinline!inotify_event!
structure contains the name of the file that changed, which is compared
to two cases: if the source \lstinline!go.c! has changed, recompile it;
if the library \lstinline!go.so! has changed, reload it. Hopefully the
first case triggers the second case, otherwise the live coder should
check the terminal for compilation error messages.

It is important to start watching before the main loop, rather than
stopping and starting repeatedly within the loop, lest events are missed
between loop iterations. To support languages other than C, it is
necessary to modify line 27 of Listing \ref{linotify} to reference the
new source file, and add the approriate compilation commands to the
\lstinline!Makefile!.

\begin{lstlisting}[language=C, caption={Using inotify}, label=linotify]
int main() {
 // ...
 int ino = inotify_init();
 if (ino == -1) return 1;
 int wd = inotify_add_watch(ino, ".", IN_CLOSE_WRITE);
 if (wd == -1) return 1;
 ssize_t buf_bytes = sizeof(struct inotify_event) + NAME_MAX + 1;
 char *buf = malloc(buf_bytes);
 while (true) {
  load_library(copy_library());
  // read events (blocking)
  bool library_changed = false;
  while (! library_changed) {
   memset(buf, 0, buf_bytes);
   ssize_t r = read(ino, buf, buf_bytes);
   if (r == -1)
    sleep(1);
   else {
    char *bufp = buf;
    while (bufp < buf + r) {
     struct inotify_event *ev = (struct inotify_event *) bufp;
     bufp += sizeof(struct inotify_event) + ev->len;
     if (ev->mask & IN_CLOSE_WRITE) {
      if (0 == strcmp("go.so", ev->name))
       // reload
       library_changed = true;
      if (0 == strcmp("go.c", ev->name))
       // recompile
       system("make --quiet");
     }
    }
   }
  }
 }
 close(ino);
 return 0;
}
\end{lstlisting}

\subsection{Squash denormals}\label{squash-denormals}

A common problem in audio processing on \lstinline!x86_64! architectures
is the large increased cost of processing denormal floating point
numbers when recursive algorithms decay to zero. This is achieved in
clive by setting the \lstinline!DAZ! (denormals-are-zero) and
\lstinline!FTZ! (flush-to-zero) CPU flags~\cite{DAZFTZ}, and ensuring
that the shared library code is compiled with the \lstinline!gcc! flag
\lstinline!-mfpmath=sse! which makes calculations use the SSE unit.

\begin{lstlisting}[language=C, caption={Setting DAZ and FTZ}, label=ldenormal]
int process(jack_nframes_t n, void *arg) {
#ifdef __x86_64__
 fenv_t fe;
 fegetenv(&fe);
 unsigned int old_mxcsr = fe.__mxcsr;
 fe.__mxcsr |= 0x8040;
 fesetenv(&fe);
#endif
 // ...
#ifdef __x86_64__
 fe.__mxcsr = old_mxcsr;
 fesetenv(&fe);
#endif
 return 0;
}
\end{lstlisting}

\section{Example}\label{example}

Thus far in this paper all the code presented has been from the live
engine. To remedy this, here is a short callback example, presenting
a simple metronome tone:

\begin{lstlisting}[language=C, caption={Metronome example}, label=lexample]
#include <math.h>
#define SR 48000
#define PI 3.141592653589793
double wrap(double x) {
  return x - floor(x);
}
typedef struct { double phase; } PHASOR;
double phasor(PHASOR *s, double hz) {
  return s->phase = wrap(s->phase + hz / SR);
}
typedef struct {
  int reloaded;
  PHASOR beat;
  PHASOR tone;
} S;
extern void go(S *s, float *in, int nin, float *out, int nout) {
  double beat = phasor(&s->beat, 120 / 60.0) < 0.25;
  double tone = sin(2 * PI * phasor(&s->tone, 1000));
  double metronome = beat * tone;
  for (int c = 0; c < nout; ++c)
    out[c] = metronome;
}
\end{lstlisting}

\section{Evaluation and future work}\label{evaluation}

Clive's code swapping is quantized to JACK process callback block boundaries,
which is not musically useful.  A workaround can be to scatter the callback
code with sample-and-hold unit generators so that the previous values are kept
until some future trigger (from an already-running clock oscillator perhaps)
synchronizes them with the desired sound.  The need in Clive to specify unit
generator storage separately from the invocation of the generator makes this
doubly painful.

Clive processes one sample at a time, which is not necessarily the most
efficient way to process audio.  It may be useful to add unit generators to
support reblocking, alongside vector operations, to Clive's library.  This
reblocking would be necessary if FFT operations were to be supported,
which would also benefit from reblocking with overlap.

The Bela~\cite{BELA} platform is interesting for low-latency audio processing.
Clive could be adapted to cross-compile into a device directory mounted on
the host, with the engine split into two parts (device to reload library
code, and host to recompile source code).  Bela also supports electronic
connections (both analogue and digital pins), which would be nice to expose
to Clive callbacks.

The OpenGL graphics library is interesting for programmable shaders.  Clive
could be adapted to load and compile GLSL source code for live-coding visuals,
perhaps focusing on fragment shaders similar to Fragmentarium~\cite{FRAG}.
A way for Clive's audio processing callbacks to send control data to OpenGL
visualisations would allow tight audio-visual synchronisation.


\section{Conclusion}\label{conclusion}

A live coding system for audio processing in C has been presented. The
author has performed with it many times since its first version in 2011,
including both seated events and club-style Algorave situations, as well
as sessions at home.

The code and some session audio+diff-casts can be found at
\begin{itemize}
\item \url{https://code.mathr.co.uk/clive}
\item \url{https://mathr.co.uk/clive}
\end{itemize}
respectively.

\bibliographystyle{acl}
\bibliography{paper}

\end{document}
